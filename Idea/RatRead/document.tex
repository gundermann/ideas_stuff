\documentclass[a4paper]{scrartcl}
\usepackage[ngerman]{babel}
\usepackage[ansinew]{inputenc} 
\usepackage{lmodern}
\usepackage[margin=2.5cm]{geometry}
\usepackage{graphicx}
\usepackage{listings} 
\usepackage{multirow}
\usepackage{cite}
\lstset{language=command.com} 
\lstset{basicstyle=\scriptsize}
\lstset{numbers=left, numberstyle=\tiny, numbersep=2pt} 

\graphicspath{{Bilder/}}

\linespread{1.5}

\title{Konzept zur Spiel-App zur Verbesserung des rationellen Lesen}
\author{Niels Gundermann}

\begin{document}
\bibliographystyle{geralpha}
\large
\maketitle
\pagebreak
\tableofcontents
\pagebreak

\listoffigures
\pagebreak

\section{Umriss}
Die App stellt ein Spiel dar, das in der Lage ist bestimmte F�higkeiten, die 
einem das Lesen erleichter, zu verbessern.\\
Zu diesen F�higkeiten geh�ren:\\
\begin{enumerate}
	\item Blickspannweite
	\item intuitive Worterkennung
	\item Behaltensleistung
\end{enumerate}
\end{document}


