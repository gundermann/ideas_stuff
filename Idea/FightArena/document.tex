\documentclass[a4paper]{scrartcl}
\usepackage[ngerman]{babel}
\usepackage[ansinew]{inputenc}
\usepackage{lmodern}
\usepackage[margin=2.5cm]{geometry}
\usepackage{graphicx}
\usepackage{listings} 
\usepackage{multirow}
\usepackage{cite}
\lstset{language=command.com} 
\lstset{basicstyle=\scriptsize}
\lstset{numbers=left, numberstyle=\tiny, numbersep=2pt} 

\graphicspath{{Bilder/}}

\linespread{1.5}

\title{Konzept zum Ego-Fighter: Fight-Arena}
\author{Niels Gundermann}

\begin{document}
\bibliographystyle{geralpha}
\large
\maketitle
\pagebreak
\tableofcontents
\pagebreak

\listoffigures
\pagebreak

\section{Umriss}
In dem Spiel geht es darum, seine Feinde in einer Arena zu besiegen
(Grundspiel). Zudem wird es m�glich sein, in einem Editor neue Arenen nach
eigenem Bild zu erstellen.\\
Ein Adventure-Mode wird es erm�glichen den Charackter in verschiedene Abenteuer
zu schicken. Die Abenteuer werden leicht konfigurierbar sein. Somit ist es
wiederum in einem Editor m�glich, diese Abenteuer selbst zu entwerfen und
anderen zur Verf�gung zu stellen. Die Erstellung des Abenteuers gliedert sich in
zwei Abschnitte.\\
Der Erste Abschnitt betrifft die Welt, in der sich der Spieler bewegen wird.
Hierbei k�nnen auch bestehende Welter genutzt werden. Dieser Abschnitt bildet
nicht nur die Umgebung ab, sondern auch s�mtliche Characktere mit denen der
Spieler interagieren kann, um sein Zeil zu erreichen. \\
Der Zweite Abschnitt bildet das Abenteuer ab. Hier wird festgelegt, welche
Aufgaben zu erf�llen sind, um das Abenteuer abzuschlie�en. Auch die St�rke und
F�higkeiten der Gegner werden hier festgelegt.\\
Letztendlich m�ssen beide Abschnitte jedoch auf einander abgestimmt werden, um
ein halbwegs sinnvolles Abenteuer zu gestalten.
\end{document}


